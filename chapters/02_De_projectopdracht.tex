\section{De projectopdracht}
\subsection{Project naam}
Voor dit project, Project Windpark Op Zee is het bedrijf Cleaneco in dienst genomen door Worley. 

\subsection{Probleem}
De Nederlandse overheid moet voldoen aan het \gls{energie akkoord}\cite{energieakkoord}, dat is vastgelegd op 6 september 2013. Dit akkoord stelt dat tegen 2023 een uitbreiding moet zijn gerealiseerd tot een operationeel windvermogen op zee van 4450 MW. De al bestaande parken en de geplande projecten bedragen samen dan ongeveer 1000 MW.

% De Nederlandse overheid moet voldoen aan het \gls{energie akkoord} vastgelegd op 6 september 2013.
% Hierin staat aangegeven dat in 2023 een opschaling moet hebben plaatsgevonden voor een operationeel windvermogen op zee van 4450MW. De reeds bestaande parken en hetgeen reeds in de pijplijn zit, tellen op tot circa 1000MW.
% Hoe kan de Nederlandse overheid de groei van windenergie op zee naar 4450 MW operationeel vermogen in 2023 bevorderen, met focus op kostenreductie, technologische vooruitgang, ruimtelijke planning en het overwinnen van belemmeringen, om zo aan de eisen te voldoen van het \gls{energie akkoord} opgesteld in september 2023?

\subsection{Doelstelling en Resultaat}
Onder opdracht van Worley is het aan Cleaneco om een technisch ontwerp te maken van een windturbinepark op zee. Hiervoor moet ook een onderhoudsplan worden opgesteld voor de komende 25 jaar. Het technisch ontwerp en onderhoudsplan zullen in de vorm van een adviesrapport aan de opdrachtgever gepresenteerd worden.

Uiterlijk in 2023 wordt een operationeel windvermogen van 4450 MW op zee gerealiseerd, in overeenkomst met de eisen van het \gls{energie akkoord}\cite{energieakkoord} van september 2023. Dit wordt bereikt met meetbare kostenverlaging, technologische vooruitgang, doelgerichte ruimtelijke planning en het succesvol aanpakken van belemmeringen. Dit project heeft significante maatschappelijke relevantie, het draagt namelijk bij aan de verduurzaming van de energieopwekking. Naast de naar schatting vijfmiljoen\cite{energieakkoord} huishoudens die zullen profiteren van de duurzame energievoorziening, zal dit project bijdragen aan het vergroten van de kennis en ervaring op het front van windparken op zee.  

\subsection{Eisen}
Er moet aan de volgende eisen worden voldaan:
\begin{itemize}
   \item Er moet onderzoek gedaan worden naar nodige producten/materialen zoals: windturbines, kabels en transformator boxen. Hierbij moeten de energieopbrengsten voor twee verschillende turbines worden onderzocht.
   \item Er moeten twee verschillende ontwerpen komen voor de positionering van de windturbines. 
   \item Er moet een operationeel windvermogen op zee van 4450MW opgeleverd worden.
   \item Er moet een onderhoudsplan meegeleverd worden voor de komende 25 jaar.
   \item Alles moet in een adviesrapport aan de opdrachtgever Worley gepresenteerd worden. 
 \end{itemize}

\subsection{Probleem oplossing}
Cleaneco zal een ontwerprapport en onderhoudsplan leveren aan Worley. Het rapport beantwoordt de tender voor het \gls{offshore} windturbinepark. Het onderhoudsplan waarborgt 25 jaar operationeel zijn. Bij projectvoltooiing volgt een adviesrapport. Om dit te verwezenlijken zal onderzoek worden gedaan naar de punten genoemd onder 2.4 Eisen. 

Het type onderzoek zal vallen onder het zogenoemde, bronnenonderzoek. De bronnen zullen onder andere komen van de database van de Haagse Hogeschool en professionals in het betreffende vakgebied.
