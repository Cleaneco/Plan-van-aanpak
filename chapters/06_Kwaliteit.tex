\section{Kwaliteit}
Er zullen meerdere maatregelen worden genomen om de kwaliteit van de producten te waarborgen. Voor alle producten geldt dat er een eindredactie zal plaatsvinden. Bij de eindredactie zullen de documentaties worden gecontroleerd op inhoudelijke en verslagtechnische correctheid. Ook zullen voor alle producten alleen betrouwbare bronnen geraadpleegd worden met relevante informatie en actuele autoriteit. 

\subsection{Planning en taakverdelingen}
Naast een eindredactie wordt om de kwaliteit van de producten en de timing van de levering hiervan te garanderen voor een goede organisatie gezorgd. Hieronder vallen een goede taakverdeling, planning en communicatie. Voor de taakverdeling en planning wordt het platform Github gebruikt, hierop zijn de taken per projectlid ingedeeld en ingepland. Zo kan worden verzekerd dat iedereen zich aan alle deadlines houdt. Als dit niet het geval blijkt te zijn, wordt dit besproken en zijn er consequenties. Naast Github is de planning ook in grote lijnen uitgewerkt in het plan van aanpak. 

\subsection{Onderzoeksvragen}
Met als doel de structuur en kwaliteit van de documentaties te waarborgen, zal de probleemstelling worden opgedeeld in kleinere problemen. Deze problemen zullen worden uitgewerkt met behulp van onderzoeksvragen.\cite{projecthandleiding}
De hoofdvraag voor het gehele project is: "Hoe kunnen de eisen volgens het energieakkoord \cite{energieakkoord} van 2023 behaald worden met behulp van een windturbinepark op kavel VI en VII?". De hoofdvraag zal worden beantwoord door meerdere deelvragen. Voor tussenproduct 1, het park ontwerp, zullen de volgende vragen beantwoord worden\cite{projecthandleiding}:
\begin{itemize}
    \item Welke twee type windturbines zijn het meest geschikt voor het windpark en waarom?
    \item Wat is de beste positionering van de windturbines voor optimale opbrengst, efficiënte bekabeling en makkelijk onderhoud?  
    \item Welke kabels zijn geschikt voor het transporteren van de energie van turbines tot het hoogspanningsstation met de verwachte stromen?
    \item Hoe ziet de bekabeling van het windpark eruit en waarom?
    \item Welke is de beste soort spanning voor dit windturbinepark, gelijk- of wisselspanning en welke effecten heeft dit op het park?
    \item Welk invloed hebben de gemaakte keuzes op de onderhoud van het park? 
\end{itemize}

Ter beantwoording van tussenproduct 2, het onderhoudsplan, zijn de volgende deelvragen opgesteld:
\begin{itemize}
    \item Hoe wordt de conditie van de componenten waaruit het windturbinepark bestaat bewaakt? 
    \item Welk onderhoud moet er plaatsvinden aan het windturbinepark?
    \item Hoe frequent moet er onderhoud plaatsvinden en hoe is dit bepaald?
    \item Hoe ziet de onderhoudsstrategie eruit? 
    \item Welke materiële en financiële risico's zijn er bij het plegen van onderhoud?
\end{itemize}

Deze vragen uitgebreid beantwoorden garandeert dat alle belangrijke aspecten voor dit project behandeld worden. 

% Tijdens het maken van de (tussen)producten zal er telkens gekeken worden naar de eisen waaraan het moet voldoen en hier naartoe worden gewerkt. Mocht iets niet lukken, dan zal er hulp worden ingeschakeld van de technisch adviseur (Ad van den Bergh).

% Gedurende het realisatie proces zullen er ook regelmatig vragen worden gesteld aan de opdrachtgever om er zeker van te zijn dat het product naar wens wordt ontworpen en gemaakt.

% Een andere manier waarop de kwaliteit wordt gewaarborgd, is door het gebruik van goede programma’s. De programma’s die gebruikt worden zijn met name, Arduino IDE en KiCad. Alle code zal in de Arduino IDE omgeving geschreven worden. Dit zorgt ervoor dat de code direct op het Arduino bordje geüpload en getest kan worden.

NOTE: Idee om te verwijzen naar bronnen en moeilijke woorden om zo de kwaliteit van het verslag te verhogen