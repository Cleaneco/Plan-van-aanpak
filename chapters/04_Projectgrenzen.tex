\section{Projectgrenzen}
\subsection{Wat wel}
\begin{itemize}
    \item Wij schrijven een plan van aanpak.
    \item Wij doen onderzoek naar de technische aspecten en de energie opbrengsten.
    \item Wjj doen onderzoek naar de verschillende windturbines, transformatoren, bekabeling en toebehoren.
    \item Wij schrijven een onderhoudsplan voor 25 jaar waarbij we gebruik maken van levensduur en statistische gegevens.
    \item Wij maken een parkontwerp waarbij rekening wordt gehouden met een optimale plaatsing van de windturbines.
    \item Wij presenteren ons parkontwerp en onderhoudsplan aan onze opdrachtgever: Worley.
\end{itemize}
\subsection{Wat niet}
\begin{itemize}
    \item Wij vragen geen vergunning aan voor een \gls{offshore} windpark op zee.
    \item Wij vragen geen subsidie aan bij de Nederlandse overheid.
    \item Wij beheren geen elektrische infrastructuur (wordt gedaan door TenneT\cite{energieakkoord}).
    \item Wij realiseren geen \gls{offshore} windpark op zee.
    \item Wij zijn niet verantwoordelijk voor oplopende problemen tijdens de realisatie fase van het \gls{offshore} windpark op zee.
\end{itemize}
\subsection{Verwachte tijdsduur}
Op dinsdag 29 augustus zijn we begonnen aan het project \textit{Duurzame Energie} en we plannen om hier 17 weken aan te besteden. In week 19 staat onze eindpresentatie gepland voor onze opdrachtgever Worley, die zal plaatsvinden op hun kantoor in Den Haag. 
We schatten in dat we gedurende dit project ongeveer 16 x 2 contacturen en 16 x 6 zelfstudie-uren zullen besteden\cite{studiewijzer}.

% De projectleden zullen bronnenonderzoek doen naar de verschillende opties voor windturbines, bekabeling 



% Het assembleren van de auto wordt door de projectleden gedaan. De nadruk ligt op de software schrijven voor de auto. Er wordt software geschreven in de Arduino IDE voor de IR sensoren, ultrasonic sensoren en het aansturen van de DC motoren en servo’s.
% \\\\
% Daarna moeten de verschillende codes worden samengebracht tot één code. Hiermee moet worden gerealiseerd dat de auto bepaalde taken uitvoert. De taken die de auto uitvoert zorgen voor twee modi, een prooi-modus om zo lang mogelijk uit handen te blijven van een andere auto en een jager-modus om een andere auto zo snel mogelijk te "vangen".
% \\\\
% Het testen en kalibreren van de IR sensoren, sonar sensoren, DC motoren, servo motoren, Arduino Mega 2560, motorshield en eventuele bijhorende printplaten/hardware wordt door de projectleden gedaan. Hiermee kan worden vastgesteld dat de bijbehorende componenten juist werken. Daarnaast kunnen er meer sensoren worden toegevoegd, zodat de auto zijn omgeving beter kan waarnemen. 

% Er kunnen ook kleine aanpassingen worden gemaakt aan de chassis van de auto. Verder kan er een printplaat worden ontworpen voor op de Arduino Mega 2560 voor het makkelijker aansluiten van connectoren.
% Het ontwerpen van de totale chassis, de sensoren en de motoren wordt niet door de projectleden gedaan.
% \\\\
% Het project is gestart op 7-2-2023 en duurt tot het einde van semester 2.

% Voor het slagen van het project wordt verwacht dat elk groepslid elke week serieus werkt aan de auto. Daarnaast besteedt ieder groepslid elke week ±8 uur aan het project. Ook houden zij zich aan de afspraken en de deadlines. Ten slotte moet er (persoonlijk) gedocumenteerd worden wat elk groepslid gedaan heeft.
