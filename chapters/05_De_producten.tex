\section{De producten}
\subsection{Verslag}
Aan het einde van semester 2 zal projectgroep Infra Vroom een verslag inleveren. In dit verslag zal duidelijk gemaakt worden wat er gedaan is en waarom. Ook zal er worden aangegeven welke problemen er waren en hoe deze zijn opgelost.
\subsection{Meetrapport}
In dit rapport worden de gedane metingen genoteerd. Voorbeelden hiervan zijn: verbruik van de motoren en afstanden van de sensoren.
\subsection{Tekening}
Er zijn een aantal tekeningen gemaakt. In deze tekeningen zijn de connecties uitgetekend. Dit zodat alle verbindingen op een efficiënte wijze kunnen worden gemaakt. In de toekomst kan er een \textit{Arduino Mega Hat} worden ontwerpen met \textit{FFC connectoren} voor nog betere verbindingen.
Ook zijn er \textit{aerodynamische} componenten toegevoegd. Zo kan genoeg grip behouden worden op het grondoppervlak om de wegligging te verbeteren.
\subsection{Planning en taakverdelingen}
Voor de taakverdeling is het platform Github gebruikt, hierop zijn de taken per projectlid ingedeeld en ingepland. Zo kan worden verzekerd dat iedereen zich aan alle deadlines houdt. Als dit niet het geval blijkt te zijn, dan wordt dit besproken en zijn er consequenties.
