\section{Project-activiteiten}
Gedurende de loop van het project zijn enkele tussen rapportage momenten waarvoor doelstellingen zijn gesteld waar naartoe zal moeten worden gewerkt. Voor elk rapportage moment zullen de gevraagde en benodigde documenten geleverd worden.

\subsection{Tussen rapportage 1}
In week 3 zal voor tussen rapportage moment 1 het plan van aanpak opgeleverd worden, deze zal ook worden gepresenteerd in de vorm van een korte pitch van 2 minuten. Hierin zal worden besproken wat er onderzocht zal worden, hoe dit zal gebeuren, en welke acties er genomen zullen worden om het eindproduct te realiseren.

\subsection{Tussen rapportage 2}
In week 6 zullen de relevante technische aspecten worden behandeld die betrokken zijn bij de realisatie van een windpark. Dit zal plaatsvinden direct na de eerste evaluatiefase. Tijdens deze evaluatie zal de verwachte jaarlijkse energieopbrengst gepresenteerd worden voor het geval dat het windpark wordt gebouwd met de geselecteerde turbines. Diverse types windturbines zullen worden ingezet om de opbrengst te demonstreren.

\subsection{Tussen rapportage 3}
In week 12 word in het algehele technische ontwerp geëvalueerd, inclusief de turbinebekabeling. Daarnaast wordt momenteel de nadruk gelegd op de onderhoudsaspecten van het windpark. Gedetailleerd wordt aangegeven welke elementen (zoals specifieke turbineonderdelen) onderhoudskosten, inclusief de vereiste frequentie en uitvoeringsmethoden. Bij het opstellen van het onderhoudsplan wordt gebruik gemaakt van statistische gegevens met betrekking tot de levensduur van componenten.

\subsection{Tussen rapportage 4}
In week 16 zal het complete projectresultaat (het parkontwerp en onderhoudsplan) gepresenteerd worden aan de opdrachtgever Worley op het kantoor gevestigd te Den Haag.